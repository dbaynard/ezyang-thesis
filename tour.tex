%!TEX root = paper.tex
\chapter{A tutorial of \Backpack{}}
\label{sec:tour}

% PROBLEM ---> SOLUTION

In this section, we will show you how to modularize a
regular expression matcher
(we take our implementation from~\cite{Fischer:2010:PRE:1863543.1863594}),
highlighting the main
features of \Backpack{} as well as introducing some core conceptual
ideas which will be elaborated upon later in this thesis.  We'll assume
familiarity with Haskell'98 or an ML-like language but won't assume you
know anything about Cabal (Haskell's package management system) or
\OldBackpack{}~\cite{backpack}.\footnote{For a more copy-paste friendly
version of this tutorial, please refer to \url{http://blog.ezyang.com/2016/10/try-backpack-ghc-backpack/}}

\section{A simple matcher in '98}

You have been asked to implement a Haskell library that provides a
regular expression matcher.  Nodding knowingly, you shuffle through your
copy of Fischer, Huch and Wilke~\cite{Fischer:2010:PRE:1863543.1863594}
and produce the code listing in Figure~\ref{fig:matcher-haskell98}.  It
is an ordinary matcher on Haskell's \verb|String| data type
(linked lists of characters), written in vanilla
Haskell'98; you also write a small test program for it.

\begin{figure}
\begin{lstlisting}
module Regex (Reg(..), accept) where

-- | A type of regular expressions.
data Reg = Eps | Sym Char | Alt Reg Reg | Seq Reg Reg | Rep Reg

-- | Check if a regular expression 'Reg' matches a 'String'
accept :: Reg -> String -> Bool
accept Eps       u = null u
accept (Sym c)   u = u == [c]
accept (Alt p q) u = accept p u || accept q u
accept (Seq p q) u = or [accept p u1 && accept q u2 | (u1, u2) <- splits u]
accept (Rep r)   u = or [and [accept r ui | ui <- ps] | ps <- parts u]

-- | Compute all splits of the string.
splits :: String -> [(String, String)]
splits [] = [([], [])]
splits (c:cs) = ([], c:cs):[(c:s1,s2) | (s1,s2) <- splits cs]

-- | Compute all possible non-empty partitions of the string
parts :: String -> [[String]]
parts [] = [[]]
parts [c] = [[[c]]]
parts (c:cs) = concat [[(c:p):ps, [c]:p:ps] | p:ps <- parts cs]
\end{lstlisting}
\caption{Source code for a regular expression matcher from~\cite{Fischer:2010:PRE:1863543.1863594}.}
\begin{lstlisting}
module Main where

import Regex

nocs = Rep (Alt (Sym 'a') (Sym 'b'))
onec = Seq nocs (Sym 'c')
evencs = Seq (Rep (Seq onec onec)) nocs
main = print (accept evencs "acc")
\end{lstlisting}
\caption{Simple test program for the matcher.}
\label{fig:matcher-haskell98}
\end{figure}

%   \begin{itemize}
%       \item The code is organized into two \emph{modules}, each heralded
%       by the \verb|module| keyword.  Each module defines types and values
%       within a distinct namespace.  In fact, Haskell modules are used purely for
%       namespacing and do not support any sort of hierarchical organization.

%       \item The \verb|import| keyword is used to bring declarations into
%       scope from other modules; for example, \verb|Main| imports \verb|Regex|
%       to bring the constructors of \verb|Reg| and \verb|accept| into scope.
%       Every module also implicitly has an import of \verb|Prelude|, which
%       provides many commonly used types and functions (e.g., \verb|print|,
%       \verb|String|, etc).

%       \item The \verb|Regex| module comes with an \emph{export list}
%       \verb|(Reg(..), accept)|, which designates which declarations should
%       be brought into scope when this module is imported.  The ability to
%       omit declarations from an export list (e.g., \verb|splits| and \verb|parts|)
%       is one of the primary mechanism by which data abstraction is achieved in
%       Haskell: internal implementation details are not exported and thus
%       cannot be accessed by end users.
%   \end{itemize}

Overjoyed by the elegance of your regular expression matcher, you decide
to package and upload your code, so that others can install it
using a package manager like \verb|cabal-install|.  To
do this, you write a \emph{Cabal file} which records some metadata
about the package in question, as in Figure~\ref{fig:matcher-packages}.
In it, you record:

\begin{figure}
\begin{tabular}{p{0.45\textwidth} p{0.45\textwidth}}
\begin{lstlisting}[language=Cabal]
-- regex.cabal
name: regex
version: 1.0
library
    exposed-modules: Regex
    build-depends: base
\end{lstlisting}
&
\begin{lstlisting}[language=Cabal]
-- regex-program.cabal
name: regex-program
version: 1.0
executable regex-program
    main-is: Main.hs
    build-depends: base, regex
\end{lstlisting}
\end{tabular}
\caption{Cabal files for the regex and regex-program packages. The Haskell source code
of these packages is in Figure~\ref{fig:matcher-haskell98}.}
\label{fig:matcher-packages}
\end{figure}

\begin{itemize}
    \item A name (\verb|name|), so others can refer to your code,
    \item A version (\verb|version|), in case you release updates
    to your code,
    \item One or more \emph{components} (libraries, executables,
    test-suites, etc.), which define essential metadata needed
    to actually build your code.  In your library, this metadata
    includes what other packages your code needs to build
    (\verb|build-depends|) as well as a list of modules
    (\verb|exposed-modules|) which this library should provide
    to those who depend on \verb|regex|.
%   (\verb|exposed-modules| and \verb|main-is|) and specifies
%   dependencies on libraries of other packages (\verb|build-depends|).
%   A library exposes modules for other packages to use when
%   they \verb|build-depends| on the library, while an
%   executable is associated with a particular \verb|hs| file
%   which is expected to define a \verb|main| function that
%   will serve as the top-level entry of the program.\footnote{For the
%   sake of realism, each package above includes \texttt{base} in their
%   \texttt{build-depends}, to ensure that the implicitly imported
%   \texttt{Prelude} module is in scope.}
\end{itemize}
%
With a Cabal file, other users can ask the package manager to build this
library; the package manager will handle downloading and building any dependencies
before building the library itself.

%   Because most users of your program don't need the test program, you
%   create a separate package for it and give it an \verb|executable|
%   section instead; instead of exposing modules, you declare a single
%   module which defines the \verb|main| entry point of the application
%   (\verb|main-is|).

\section{Functorizing the matcher}

One day,
a user writes in: ``Your regular expression library is very nice,
but I want to use it a different string representation: \verb|ByteString|.
Can you provide a \verb|ByteString| version of your library?''
Eager to please, you consider how to go about supporting this use
case:

\begin{itemize}
    \item You could make a new library, \verb|regex-bytestring|,
    copypasting the code from your first implementation and
    modifying the imports and code until it works on \verb|ByteString|s.
    But you're not very keen on this: the core logic of regular expression
    matching is the same whether or not its \verb|String|s or
    \verb|ByteString|s, and you'd prefer not to duplicate it $n$ times
    for each string representation you want to support.

    \item Another possibility is to abstract over the string type,
    defining a record or type class to parametrize over the implementations
    of operations you might need.  However, as mentioned in Section~\ref{sec:intro},
    this has the downside of requiring you to add extra class constraints
    or dictionary arguments everywhere in your code.
\end{itemize}
%
\Backpack{} offers a third option: functorize (in the ML sense) the matcher over a
\emph{module} providing the implementation of strings.  To do this, we
create a new signature named \verb|Str|, which provides abstract string
and element types, and all the string operations our matcher requires,
and rewrite our module to make use of this module, as seen in
Figure~\ref{fig:matcher-regex-indef-source}.  We need only to
treat \verb|Str| abstractly: in the new module, the list operations
\verb|null| and \verb|[c]| are replaced with abstract variants from the
signature; furthermore, the implementations of \verb|splits| and
\verb|parts| are deferred to the signature implementation.

\begin{figure}
\begin{lstlisting}
signature Str where

data Str
data Chr
instance Eq Str

null      :: Str -> Bool
singleton :: Chr -> Str
splits    :: Str -> [(Str, Str)]
parts     :: Str -> [[Str]]
\end{lstlisting}
\caption{Source code for a signature specifying abstract strings.}

\begin{lstlisting}
module Regex where

import Prelude hiding (null)
import Str

data Reg = Eps | Sym Chr | Alt Reg Reg | Seq Reg Reg | Rep Reg

accept :: Reg -> Str -> Bool
accept Eps       u = null u
accept (Sym c)   u = u == singleton c
accept (Alt p q) u = accept p u || accept q u
accept (Seq p q) u = or [accept p u1 && accept q u2 | (u1, u2) <- splits u]
accept (Rep r)   u = or [and [accept r ui | ui <- ps] | ps <- parts u]
\end{lstlisting}
\caption{Source code for regular expression matcher, parametrized by the Str signature.}
\label{fig:matcher-regex-indef-source}
\end{figure}

Because the signature contains a full description of all the types
and functions from \verb|Str|, you can \emph{typecheck} \verb|Regex|
against the signature, even if you don't have an implementation of
\verb|Str| available.  To use the package manager to typecheck
your package, you need only modify your Cabal file so that
\verb|Str| is specified in the \verb|signatures| field, as in
Figure~\ref{fig:matcher-regex-indef-cabal}.

\begin{figure}
\begin{lstlisting}[language=Cabal]
name: regex-indef
version: 1.0
library
    exposed-modules: Regex
    signatures: Str
    build-depends: base
\end{lstlisting}
\caption{Cabal files for the regex-indef package, which provides a Regex
module parametrized by string implementation.}
\label{fig:matcher-regex-indef-cabal}
\end{figure}

\section{Instantiating the matcher}

A functorized matcher is great, but to actually run it, we must
\emph{instantiate} it with an implementation of strings.  Any implementation
of strings must provide the types and functions specified by the requirement
of the package in question. For example, Figure~\ref{fig:matcher-str-string-source}
gives the implementation of string operations on \verb|String|.

\begin{figure}
\begin{tabular}{p{0.55\textwidth} p{0.40\textwidth}}
\begin{lstlisting}
module Str.String where

import Prelude as P

type Str = String
type Chr = Char

null :: Str -> Bool
null = P.null

singleton :: Chr -> Str
singleton = (\c -> [c])

splits :: Str -> [(Str, Str)]
splits = ...

parts :: Str -> [[Str]]
parts  = ...
\end{lstlisting}
&
\begin{lstlisting}[language=Cabal]
name: str-string
version: 1.0
library
    exposed-modules: Str.String
    build-depends: base
\end{lstlisting}
\end{tabular}
\caption{An implementation of Str, alongside its package description.}
\label{fig:matcher-str-string-source}
\end{figure}

One inconvenience is that we must define monomorphic
versions of \verb|null| and \verb|singleton|, as Backpack will require
the type of an implementing function to exactly match the type
in the signature: the polymorphic function \verb|[a] -> Bool| is not
considered a valid implementation of \verb|[Char] -> Bool|.  See
Section~\ref{sec:relaxed-matching} for more about this restriction.

%   Signatures in Backpack are width-subtyped, so
%   \verb|Str.String| could also contain more exported functions and still
%   be considered to implement \verb|Str|.

%   In practice, the current convention in the Haskell community when
%   defining new modules is to pick a new name.  Thus, it's more likely
%   that the \verb|Str| module from \verb|str-string| will actually
%   be named something like \verb|Str.String|.  In this case,


In a traditional ML module system, we would then specify that we wanted this
particular implementation to be used with our functorized matcher by writing
a functor application in the module language
The problem with this is two-fold: first, if there are many parameters to a functor
and these parameters must be passed down to sub-functor applications, explicitly specifying
all of the arguments can be a chore.  Second, the module language in ML is essentially
more syntax that the compiler knows how to handle; but if we are seeking
\emph{package} level modularity, the package manager must be involved.

\Backpack{} solves the problem differently, a client instantiates a package
by manipulating the namespace of required and provided modules.
A requirement of a package is instantiated when
another module under the same name is brought into scope---this scoping
mechanism is handled already by Cabal.

In Figure~\ref{fig:matcher-functorized-packages}, we have Cabal instantiate
\verb|regex-indef| with \verb|str-string|.
Here, the executable \verb|regex-program| uses the
\verb|mixins| field to rename the module from \verb|str-string| to the
same name as \verb|regex-indef|'s signature.%
%
\footnote{Why didn't \texttt{str-string} just export
a module named \texttt{Str}?  The prevailing convention in the Haskell
community is that modules should be given unique names; naming all
implementations of the \texttt{Str} signature \texttt{Str} would violate
this convention.  At this point in time, it's not clear if the
convenience of not needing to rename modules overrides the benefits of
being able to refer to a module uniquely by its name.}

\begin{figure}
\begin{lstlisting}[language=Cabal]
name: regex-program
version: 1.0
executable regex-program
    main-is: Main.hs
    build-depends: base, regex-indef, str-string
    mixins: str-string (Str.String as Str)
\end{lstlisting}
\caption{Package descriptions for \texttt{regex-program}, which brings
Regex into scope instantiated with the \texttt{Str} implementation from \texttt{str-string}.}
\label{fig:matcher-functorized-packages}
\end{figure}

The process of wiring up requirements and provisions when they share names
is called \emph{mix-in linking}.  There are two very useful
ways to understand the process.  In Figure~\ref{fig:regex-indef-instantiated}, we
\emph{pictorially} represent each depended upon library as a block, with input and output
ports representing signatures and modules respectively.  Intuitively, the process
of mix-in linking involves ``wiring up'' these signatures and modules by matching
up names which are the same.

\begin{figure}
\center%
\includegraphics{figures/regex-indef-instantiated.pdf}
\caption{This is the wiring diagram for regex-program. The left side of blocks
(representing libraries) have input ports (required signatures), while the right hand side
of blocks have output ports (provided modules). Ports are wired to show how requirements are
filled: a kink indicates some renaming took place. Mixin linking wires up requirements
and provisions which have the same module name.}
\label{fig:regex-indef-instantiated}
\end{figure}

We can also consider an intermediate representation of the package language
\emph{after} mix-in linking,
as in Figure~\ref{fig:matcher-bkp}, where every dependency on another library
is given an \emph{explicit instantiation} specifying how all of its requirements
are filled---a \emph{syntactic} interpretation of
the package description. We can identify a library plus an instantiation by
specifying a \emph{\uid{}}.  In the case of \verb|regex-indef|, the
\uid{} indicates that the requirement \verb|Str| is to be filled
with the module \verb|Str.String| from the library \verb|str-string|.\footnote{In
fact, GHC 8.2 comes with a mode for directly parsing and compiling this intermediate
representation, which is used extensively by our test suite.}  This intermediate
representation will play an important role in managing the abstraction barrier
between the compiler and package manager, and we will repeatedly return to
it throughout the rest of this thesis.

\begin{figure}
\begin{lstlisting}
unit regex-program where
    dependency base
    dependency str-string
    dependency regex-indef[Str=str-string:Str.String]
    module Main where ...
\end{lstlisting}
\caption{The intermediate representation of the packages from Figure~\ref{fig:matcher-functorized-packages}.
The reference to \texttt{regex-indef} in \texttt{regex-program} now consists of an explicit functor application.}
\label{fig:matcher-bkp}
\end{figure}

One important practical consideration is whether or not there is any
performance cost to using signatures, deferring the implementation of
a module until later.  In particular, if one separately
compiles uninstantiated components to machine code, no cross-package
inlining can occur, since the component is compiled only against a
signature and not against the code that implements the signature.  For
Haskell and GHC, cross-module inlining is a major contributor to
performance~\cite{PeytonJones:2002:SGH:968417.968422}, so our implementation of \Backpack{} does \emph{not}
separately compile uninstantiated packages. Instead, every distinct
instantiation of a component is compiled against the code that
implements its requirements. This process is managed by Cabal to avoid
unnecessary recompilation, similarly to how Cabal avoids recompiling
dependencies that are already installed.

\section{Reusing libraries with different instantiations}

The reason we parametrized \verb|regex| was so that we could use it
with another implementation of \verb|Str|.  Figure~\ref{fig:str-bytestring}
gives a \verb|bytestring| based implementation of \verb|Str|, which shows
off another aspect of signature matching with Backpack: declarations from
a signature do not have to be implemented by the module itself: they can
be \emph{reexported} from another module.  In this example, all the declarations
defined in \verb|Data.ByteString|, which include \verb|null| and \verb|singleton|,
are reexported by the \verb|module Data.ByteString| line in the export list
of \verb|Str.ByteString|.

\begin{figure}
\begin{lstlisting}
module Str.ByteString (
    module Data.ByteString,
    Str, Chr,
    splits, parts,
) where

import Prelude hiding (length, null, splitAt)
import Data.Word
import Data.ByteString

type Str = ByteString
type Chr = Word8

splits :: Str -> [(Str, Str)]
splits s = fmap (\n -> splitAt n s) [0..length s]

parts :: Str -> [[Str]]
parts s | null s    = [[]]
        | otherwise = do n <- [1..length s]
                         let (l, r) = splitAt n s
                         fmap (l:) (parts r)
\end{lstlisting}
\begin{lstlisting}[language=Cabal]
name: str-bytestring
version: 1.0
library
    exposed-modules: Str.ByteString
    build-depends: base, bytestring
\end{lstlisting}
\caption{An implementation of Str backed by \texttt{bytestring},
and its package description.}
\label{fig:str-bytestring}
\end{figure}

We can modify \verb|regex-program| to use \verb|str-bytestring| instead
of \verb|str-string| when instantiating \verb|regex-indef| by modifying
all appropriate references in the Cabal file.  However, we can also use
both instantiations at the same time by specifying \verb|regex-indef|
twice in \verb|mixins|, as seen in the Cabal file in
Figure~\ref{fig:regex-program-multi}.
This example operates a bit differently than our previous examples:
instead of renaming the module from \verb|str-string| to be \verb|Str|,
we instead rename the \emph{requirement} from \verb|regex-indef| to
\verb|Str.String| and \verb|Str.ByteString|.  This gives us two copies
of \verb|regex-indef|: one with \verb|Str| filled with
\verb|Str.String|, and the other with \verb|Str| filled with
\verb|Str.ByteString|.  To distinguish the provided modules of each
instance, we rename them to two distinct names which \verb|Main|
can import independently.  We can visualize these two instantiations
diagramatically, as in Figure~\ref{fig:regex-indef-twice}, or
we can look at the intermediate representation of the package language
after mixin linking, as in Figure~\ref{fig:matcher-twice-bkp}.

\begin{figure}
\begin{lstlisting}[language=Cabal]
name: regex-program
version: 1.0
executable regex-program
    main-is: Main.hs
    build-depends: base, regex-indef, str-string, str-bytestring
    mixins:
        regex-indef (Regex as Regex.String) requires (Str as Str.String),
        regex-indef (Regex as Regex.ByteString) requires (Str as Str.ByteString)
\end{lstlisting}
\caption{Package descriptions for \texttt{regex-program}, which brings
Regex into scope instantiated with the \texttt{Str} implementation from \texttt{str-string}.}
\label{fig:regex-program-multi}
\end{figure}


\begin{figure}
\center\includegraphics{figures/regex-indef-twice.pdf}
\caption{Two instantiations of \texttt{regex-indef} which each define
\texttt{Reg}.  These \texttt{Reg}s are not type equivalent, because
their respective wiring diagrams are different.}
\label{fig:regex-indef-twice}
\end{figure}

\begin{figure}
\begin{lstlisting}
unit regex-program where
    dependency base
    dependency str-string
    dependency str-bytestring
    dependency regex-indef[Str=str-string:Str.String] (Regex as Regex.String)
    dependency regex-indef[Str=str-bytestring:Str.ByteString] (Regex as Regex.ByteString)
    module Main where
        import qualified Regex.String
        import qualified Regex.ByteString
        ...
\end{lstlisting}
\caption{The intermediate representation \texttt{regex-program} with two instantiations of \texttt{regex-indef}.  The \uid{}s after \texttt{dependency} form the type identity of the entities declared in \texttt{Regex.String} and \texttt{Regex.ByteString}.}
\label{fig:matcher-twice-bkp}
\end{figure}

An important subtlety arises in this situation.  Ordinarily, two types
are considered equivalent if they have the same \emph{identity}: an
identity consists of both the name of the type and the name of the
module which originally defined the type.  Previously, we could
uniquely identify a type based on what module it was defined in,
but with \Backpack{}, the identity of \verb|Reg| must somehow
depend on how we decided to implement \verb|Chr|: a \verb|Reg|
containing a \verb|Char| is very different from a \verb|Reg| containing
a \verb|Word8|.

\Backpack{} takes a conservative approach to determining the
identity of a type: it computes the type identities of all identifiers
defined in a package based on the identities of the modules that fill
the signatures of the package.  Diagramatically, the identity
incorporates the \emph{wiring diagrams} of the components which defined
the types in question; in the intermediate representation after mix-in
linking, the identity incorporates the \uid{} (which
identifies both the library and how it is instantiated).

\section{Composing libraries with requirements}

When we discussed instantiation, we stated that modules and signatures
with the same name would be linked together.  If two signatures from the
same library are brought into scope under the same name, they will also
be linked, effectively \emph{merging} the two requirements together.
For example, suppose that you had two packages \verb|p| and \verb|q|
which both were parametrized by \verb|Str|, with the first and second
signatures in Figure~\ref{fig:signature-merging}.  If we were to use
both packages together in the same third package \verb|r| (e.g.,
\verb|build-depends: p, q|), the two \verb|Str| signatures would merge
to form the third signature in Figure~\ref{fig:signature-merging}.

\begin{figure}
\begin{tabular}{p{0.30\textwidth} p{0.30\textwidth} p{0.30\textwidth}}
\begin{lstlisting}
signature Str where
  data Str
  null  :: Str -> Bool
\end{lstlisting}
&
\begin{lstlisting}
signature Str where
  data Str

  empty :: Str
\end{lstlisting}
&
\begin{lstlisting}
signature Str where
  data Str
  null  :: Str -> Bool
  empty :: Str
\end{lstlisting}
\end{tabular}
\caption{The first and second signatures merge to form the third signature.}
\label{fig:signature-merging}
\end{figure}

As before, Figure~\ref{fig:signature-merging-interp} illustrates the
pictorial and syntactic interpretations of this merging process.  It's
worth pointing out that in the syntactic interpretation, the first
occurrence of \verb|<Str>| is a binding occurrence for the subsequent
occurrences of \verb|<Str>| in the body of the unit.  The
\verb|Str| required by \verb|r| is used to instantiate the requirements
of \verb|p| and \verb|q|. However, it is not necessary to write the
signature of \verb|r| from scratch: instead, we compute it automatically
by merging the requirements of \verb|p| and \verb|q|.%
%
\footnote{In the package description, it's not necessary to declare \texttt{Str}
in the \texttt{signatures} field, as it can be inferred from the requirements
of \texttt{p} and \texttt{q}.  We render it explicitly in the intermediate
representation for clarity.}

\begin{figure}
\begin{tabular}{p{0.45\textwidth} p{0.45\textwidth}}
\center\includegraphics{figures/p-q-merge-Str.pdf}
&
\vspace{2em}
\begin{lstlisting}
unit r <Str> where
    dependency p[Str=<Str>]
    dependency q[Str=<Str>]
\end{lstlisting}
\end{tabular}
\caption{Representation of \texttt{r} as a diagram, and as the intermediate
representation after mix-in linking.}
\label{fig:signature-merging-interp}
\end{figure}

Libraries with requirements can be composed in arbitrarily complex ways;
for example, a requirement can be filled with a module from another library,
which itself has a requirement.  Because type identity relies
only on how a package is instantiated (and not when it is instantiated),
it makes no difference to the end user whether or not a library
is instantiated earlier or later.

\section{Refining types in signatures}

Suppose that you wanted to write a library of regular expressions for
matching, say, e-mails, which you wanted to build on top of
\texttt{regex-indef}.  In particular, you want to keep the particular
\emph{string} representation (\verb|Str|) abstract, but you want
to assert that, whatever string it is, it consists of Unicode
characters (\verb|Char|).  In \Backpack{}, you
write a small signature to link with the existing signature:

\begin{lstlisting}
    signature Str where
        type Chr = Char
\end{lstlisting}

\noindent
The effect of linking this signature to \verb|regex-indef| is that
all occurrences of \verb|Chr| are refined to be \verb|Char|;
for example, with this signature you can now write the regular
expressions (e.g., \verb|Rep (Alt (Sym 'a') (Sym 'b'))|) from
\verb|regex-program| or pass a \verb|Char| to \verb|singleton|.
Previously, this would not have been possible, because the
abstract type \verb|Chr| was considered distinct from \verb|Char|.

\section{Reusing and thinning signatures}

For applications more complex than a regular expression matcher, the set
of operations one might need from a string implementation will expand
correspondingly.  In such a case, there will be less code duplication if
a signature for strings is written once and then reused in packages that
need it.

A signature can be straightforwardly packaged into a library, simply
by declaring a library with one or more signatures, but no modules.
We call such packages \emph{signature packages}:

\begin{lstlisting}[language=Cabal]
    name: str-sig
    version: 1.0
    library
        signatures: Str
        build-depends: base
\end{lstlisting}

\noindent
In fact, \Backpack{} treats such signature packages specially.
Ordinarily, a required signature cannot be thinned: if you are using a
package that requires \verb|null :: Str -> Bool|, you must \emph{always}
provide that entity (since it may be used to implement some provided
function which you are making use of).  However, this is never the case
for a signature package; an entity declared in a signature here is only
ever used in the types of another package.  So \Backpack{} permits you
to thin requirements of signature packages by specifying an explicit
export list in the local signature, as long as all of the types which
are mentioned in any of the remaining requirements are also preserved.
For example, the following signature would thin any inherited signatures
from signature packages to just require \verb|Str|, \verb|null| and
\verb|empty|.

\begin{lstlisting}
    signature Str (Str, null, empty) where {- empty -}
\end{lstlisting}

%%% Local Variables:
%%% mode: latex
%%% TeX-master: "paper"
%%% End:
