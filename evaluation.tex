%!TEX root = paper.tex

\chapter{Evaluation}
\label{sec:evaluation}

% Audience:
%   - Does Backpack work?
%   - Does it solve my problems?
%   - Should I use it?
%   - Is the extension maintainable?

How can we tell if \Backpack{} works?  The proof of the pudding is
in the eating: in this chapter, we describe some of the case studies
we've implemented using \Backpack{} to help validate our implementation.

\section{Parametrizing a library}

A number of well-known packages in the Haskell ecosystem are parametric over
implementations of certain core types.  To show \Backpack{} is sufficient
to handle real world examples, we used \Backpack{} to
reparametrize \verb|unix|, \verb|tagstream-conduit| and \verb|reflex| packages, replacing
the previous strategies these packages used to support multiple
implementations of core types.  We've chosen to showcase these three
packages as they ended up being Backpack'ed in quite different ways.

\subsection{Removing copy-pasted code from unix}

The \texttt{unix}\footnote{\smaller\url{https://github.com/haskell/unix}} package provides both \verb|String| and
\verb|ByteString| versions of modules by having multiple copies of the
module with slight modifications.  For example,  here are the \verb|String|
and \verb|ByteString| versions of the \verb|getEnv| function:

\begin{lstlisting}
    -- System.Posix.Env
    getEnv :: String -> IO (Maybe String)
    getEnv name = do
        litstring <- withFilePath name c_getenv
        if litstring /= nullPtr
            then liftM Just (peekFilePath litstring)
            else return Nothing
\end{lstlisting}

\begin{lstlisting}
    -- System.Posix.Env.ByteString
    getEnv :: ByteString -> IO (Maybe ByteString)
    getEnv name = do
        litstring <- B.useAsCString name c_getenv
        if litstring /= nullPtr
            then liftM Just (B.packCString litstring)
            else return Nothing
\end{lstlisting}
%
The structure of the code between for these modules is essentially identical,
except for a few identifiers: \verb|String| versus \verb|ByteString|,
\verb|withFilePath| versus \verb|B.useAsCString|, and \verb|peekFilePath|
versus \verb|B.packCString|.

To convert this function to use \Backpack{}, we created
a signature file which covers the identifiers which vary across the
two implementations:

\begin{lstlisting}
    signature Str where
        import Foreign.C.String (CString)
        data Str
        useAsOSString :: Str -> (CString -> IO a) -> IO a
        packOSString :: CString -> IO Str
        -- ... and more for other functionality
\end{lstlisting}
%
\ldots{}and then rewrote the modules to import the signatures and use
them instead:

\begin{lstlisting}
    import Str
    getEnv :: Str -> IO (Maybe Str)
    getEnv name = do
        litstring <- withOSString name c_getenv
        if litstring /= nullPtr
            then liftM Just (packOSString litstring)
            else return Nothing
\end{lstlisting}
%
There were five modules for which there were both \verb|String| and
\verb|ByteString| variants. We divided these modules into three packages based
on what functionality they needed:  \verb|unix-env| required \verb|Str|,
\verb|unix-fs| required \verb|FilePath|, and \verb|unix-process|, which
required both \verb|Str| and \verb|FilePath|.  We also added a new
package \verb|unix-common| to store functionality that did not depend
on strings or filepaths specifically.

Consequently, we were able to delete the five duplicates of these
modules---removing 954 lines of code including comments and white space.
The total savings are a bit less, because we added source code for the
signatures, implementations of the signatures and new Cabal files for
each of the new packages.  However, the signatures and implementations
can be reused over multiple packages, and the resulting packages are
\emph{more} flexible, in the sense that they can be instantiated with
any string implementation an end user wants.

\subsection{Removing dictionary-passing style from tagstream-conduit}

The \texttt{tagstream-conduit}
package\footnote{\smaller\url{https://hackage.haskell.org/package/tagstream-conduit}}
is an HTML tag parser that supports both \verb|Text| and
\verb|ByteString|, using a record of functions to parametrize over key
string manipulation functions.  For example, here is the implementation
of \verb|decodeEntitiesText|, which operates on the \verb|Text| type:

\begin{lstlisting}
-- | Decode the HTML entities e.g. @&amp;@ in some text into @&@.
decodeEntitiesText :: Monad m => Conduit Token m Token
decodeEntitiesText =
  decodeEntities
    Dec { decToS     = L.toStrict . B.toLazyText
        , decBreak   = T.break
        , decBuilder = B.fromText
        , decDrop    = T.drop
        , decEntity  = decodeEntity
        , decUncons  = T.uncons }
  where decodeEntity entity =
          CL.sourceList ["&",entity,";"]
          $= XML.parseText def { XML.psDecodeEntities = XML.decodeHtmlEntities }
          $= CL.map snd
          $$ XML.content
\end{lstlisting}
%
\verb|decodeEntities| is a parametric function which takes the
\verb|Dec| structure as an argument.  Its type signature is:

\begin{lstlisting}
    decodeEntities :: (Monad m, Monoid builder, Monoid string,
                       IsString string, Eq string)
                   => Dec builder string
                   -> Conduit (Token' string) m (Token' string)
\end{lstlisting}
%
\verb|Dec| in turn is a structure
that contains implementations for each of the operations:

\begin{lstlisting}
    -- | A decoder.
    data Dec builder string = Dec
      { decToS     :: builder -> string
      , decBreak   :: (Char -> Bool) -> string -> (string,string)
      , decBuilder :: string -> builder
      , decDrop    :: Int -> string -> string
      , decEntity  :: string -> Maybe string
      , decUncons  :: string -> Maybe (Char,string)
      }
\end{lstlisting}
%
In \Backpack{}, we monomorphized \verb|decodeEntities|, replacing
it with an implementation that utilized signatures for strings:

\begin{lstlisting}
    signature Str where
        import Data.String

        data Str

        instance Monoid Str
        instance IsString Str
        instance Eq Str

        drop         :: Int -> Str -> Str
        decodeEntity :: Str -> Maybe Str
        uncons       :: Str -> Maybe (Char, Str)
        break        :: (Char -> Bool) -> Str -> (Str, Str)
\end{lstlisting}
%
\ldots{}and string builders:

\begin{lstlisting}
    signature Builder where
        import Str

        data Builder
        instance Monoid Builder

        builderToStr :: Builder -> Str
        strToBuilder :: Str -> Builder
\end{lstlisting}
%
Additionally, the \verb|Text| and \verb|ByteString|
modules also had some copy-pasted code for miscellaneous parsing
functionality; we extended the string signature with seven more
functions and added a signature for parsers:

\begin{lstlisting}
    signature Parser where

    import Control.Applicative
    import Str

    data Parser a
    instance Functor Parser
    instance Applicative Parser
    instance Monad Parser
    instance Alternative Parser

    anyChar :: Parser Char
    takeTill :: (Char -> Bool) -> Parser Str
    char :: Char -> Parser Char
    satisfy :: (Char -> Bool) -> Parser Char
    string :: Str -> Parser Str
    skipSpace :: Parser ()
    takeRest :: Parser Str
    parseOnly :: Parser a -> Str -> Either String a
\end{lstlisting}
%
With this refactor, we removed 258 lines of duplicated implementation
code and also eliminated the indirection in the implementation
of \verb|decodeEntities|, making it possible for GHC to inline
these functions---in the old version of the code, this would only
be possible if \verb|decodeEntities| was inlined at its call-sites
in \verb|decodeEntitiesText| and \verb|decodeEntitiesBS|.  (GHC
may not do this if \verb|decodeEntities| is a large function.)

\subsection{Removing type class indirection from Reflex}

Reflex is a framework for high-performance functional reactive
programming.  It has two implementations of its core ``timeline'' (the core
engine responsible for scheduling and dispatching events):
a pure but inefficient reference implementation, and a fast
Spider timeline based on mutable state.  Most real world users
will only make use of the Spider timeline, but the test suite
uses the reference implementation to check the correctness of the
more complicated Spider timeline.

Ordinarily, these implementations are abstracted over the \verb|Reflex|
type class:

\begin{lstlisting}
    class ( MonadHold t (PushM t)
          , MonadSample t (PullM t)
          , MonadFix (PushM t)
          , Functor (Dynamic t)
          , Applicative (Dynamic t)
          , Monad (Dynamic t)
          ) => Reflex t where
      data Behavior t :: * -> *
      data Event t :: * -> *
      data Dynamic t :: * -> *
      data Incremental t :: * -> *
      type PushM t :: * -> *
      type PullM t :: * -> *
      pull :: PullM t a -> Behavior t a
      -- 19 more methods
\end{lstlisting}
%
Compiled na\"\i{}vely, code that makes use of this type class know nothing
about the underlying implementation of these methods, thwarting GHC's
inliner and optimizer. This is a big loss, since most of the time you
only needed the Spider timeline implementation.  In fact, even when
the implementation is statically known via the type a method is used
at, GHC will still occasionally fail to specialize\footnote{\url{https://ghc.haskell.org/trac/ghc/ticket/12791}
and \url{https://mail.haskell.org/pipermail/ghc-devs/2016-October/013170.html}},
causing a huge slowdown.

With Reflex, we investigated using \Backpack{} to eliminate all uses of
the \verb|Reflex| type class, thus eliminating dictionaries altogether.
Backpack'ing Reflex posed some unique problems:

\begin{itemize}
    \item The Spider timeline's type is not \verb|SpiderTimeline|
        but \verb|SpiderTimeline x|---in particular, it takes a phantom
        type parameter similar to the \verb|s| of the \verb|ST|
        monad~\cite{Launchbury:1994:LFS:773473.178246} that is used to
        distinguish between different running instances of the Spider
        timeline.

        This is not a problem for the \verb|Reflex| type class:
        if we declare an instance of \verb|Reflex| for \verb|SpiderTimeline x|
        we get the following type signature for the method \verb|pull|:

        \begin{lstlisting}
        pull :: PullM (SpiderTimeline x) a -> Behavior (SpiderTimeline x) a\end{lstlisting}
        Here, the signature states that if we call pull on a \verb|PullM|
        from a Spider timeline, we will get a \verb|Behavior|
        from the very same Spider timeline.

        However, if in \Backpack{} we existentially quantify
        the type variable \verb|x| to create a well formed
        top-level type declaration, we will end up with the
        following types:

        \begin{lstlisting}
        data Timeline = forall x. SpiderTimeline x
        pull :: PullM Timeline a -> Behavior Timeline a\end{lstlisting}
        This doesn't have the intended semantics: here, \verb|pull|
        accepts a \verb|PullM| from some timeline, and then gives
        you a \verb|Behavior| from another timeline, but doesn't tell
        you if they are the same or not, making it impossible to tell
        statically if you are mixing up two different timelines or not.

    \item Unlike the previous two case studies, we wanted to build
        the \Backpack{} interfaces on top of the existing Reflex
        library without modifying it at all, so that clients could
        choose between the old-style type-class based interface, or
        the new-style \Backpack{} interface.  This meant, as much
        as possible, we wanted to reuse types and classes from
        Reflex, as long as they did not depend on the \verb|Reflex|
        type class.  (We can't reuse declarations which depend
        on \verb|Reflex|, since our intention is to specialize the
        type class away.)

%   \item Reflex defines a large number of type classes which mention
%       the associated types of \verb|Reflex|.  These type classes
%       cannot be easily eliminated via \Backpack{} (classes like
%       \verb|MonadHold| and \verb|MonadSample| are for allowing
%       users to use functions on arbitrary monad transformer stacks),
%       but must be defined
\end{itemize}
%
Here is an excerpt for the signature we wrote for Reflex:

\begin{lstlisting}
    data Timeline t
    class HasTimeline t

    type Event          t = C.Event         (Timeline t)
    type Dynamic        t = C.Dynamic       (Timeline t)
    type Behavior       t = C.Behavior      (Timeline t)
    type Incremental    t = C.Incremental   (Timeline t)
    type EventSelector  t = C.EventSelector (Timeline t)

    data PushM t a
    data PullM t a

    instance HasTimeline t => MonadHold     (Timeline t) (PushM t)
    instance HasTimeline t => MonadSample   (Timeline t) (PullM t)
    instance HasTimeline t => MonadFix      (PushM t)
    instance HasTimeline t => Functor       (Dynamic t)
    instance HasTimeline t => Applicative   (Dynamic t)
    instance HasTimeline t => Monad         (Dynamic t)
    -- Instances for the superclasses

    pull :: HasTimeline t => PullM t a -> Behavior t a
\end{lstlisting}
%
There are a number of things to point out about this signature:

\begin{itemize}
    \item The abstract type representing timelines is
        \verb|data Timeline t|, taking a phantom type parameter
        which will be used to track timelines.
        Reflex also associates this phantom type parameter with
        a class constraint \verb|HasSpiderTimeline|, so we need
        an abstract class \verb|HasTimeline| to abstract over
        this extra constraint (we can implement this class using
        a constraint synonym).

    \item Rather than define abstract types for \verb|Event|,
        \verb|Dynamic| and the other associated data families,
        we instead require them to be type synonyms of the
        appropriate data family for our abstract type \verb|Timeline|.
        The benefit of doing things this way is that we can
        reuse the existing \verb|MonadHold| and \verb|MonadSample|
        type classes which don't use the \verb|Reflex|
        type class directly, but do refer to the associated
        data families it defines.

        In contrast, the type synonym families get abstract
        types, because we need to declare instances on them,
        and Haskell unconditionally forbids type family applications in
        instances.\footnote{GHC Haskell should relax its rule
        to forbid type family applications which contain free
        variables, which would be sufficient to avoid undecidability
        of instance resolution.  See also
        \url{https://ghc.haskell.org/trac/ghc/ticket/13262}}

    \item In a signature, a user is required to declare
        instances not only for any classes they want to support,
        but also all of the superclasses of those classes.  For
        brevity, we've
        omitted those superclasses.

    \item Inside the module parametrized over this signature,
        we defined some instances which overlapped with existing
        instances from the Reflex library.  For example, Reflex
        defines an instance \verb|instance Reflex t => Behavior t|,
        which makes use of the \verb|Reflex| type class to provide
        operations,
        while we define \verb|instance Behavior (Timeline t)|, which
        directly makes use of operations from the signature.
        Fortunately, the second specialized instance is more specific
        than the first instance, so it is preferred during instance
        resolution.
\end{itemize}
%
Reflex also has a separate interface, \verb|ReflexHost|, for listening
to events, which is only supported by the Spider implementation.
To accommodate this, we created another library which mixed in
the basic Reflex functionality and extended the signature with the
extra requirements that were needed.

\section{Replacing build-depends with signatures}

One of the future applications we envision for \Backpack{} is subsuming
traditional version bounds in packages.  To show that \Backpack{} is up
to the task, we took a few packages in the public Hackage repository and
rewrote them to deprecate some of their \verb|build-depends| in favor of
signatures instead:

\begin{itemize}
    \item We rewrote \texttt{binary-0.8.0.0}\footnote{\smaller\url{https://hackage.haskell.org/package/binary-0.8.0.0}}
          to have signatures
          for \texttt{byte\-string}, \texttt{containers} and \texttt{array},
          demonstrating that GHC's core libraries can be modularized
          over. (23 signatures)

    \item We rewrote \texttt{ghc-simple-0.3}\footnote{\smaller\url{https://hackage.haskell.org/package/ghc-simple-0.3}} to have signatures
          for \texttt{ghc}, demonstrating that the GHC API (a very
          complicated API) can be modularized over. (57 signatures)
\end{itemize}
%
For example, here is the signature we wrote for
\texttt{Data.\allowbreak{}Array.\allowbreak{}IArray} in \texttt{binary},
exercising many of Haskell's features including type classes:

\begin{verbatim}
    {-# LANGUAGE RoleAnnotations #-}
    {-# LANGUAGE FlexibleInstances #-}
    {-# LANGUAGE MultiParamTypeClasses #-}
    {-# LANGUAGE KindSignatures #-}
    {-# LANGUAGE ExplicitForAll #-}
    module Data.Array.IArray (
        module Data.Array.IArray,
        module Data.Ix
        ) where

    import Data.Ix

    type role Array nominal representational
    data Array i e
    class IArray (a :: * -> * -> *) e
    instance IArray Array e
    bounds :: (IArray a e) => forall i. Ix i => a i e -> (i, i)
\end{verbatim}
%
Adding signatures to packages is a straightforward (if a little tedious)
process.  However, we made two interesting observations.

First, there is a certain amount of human judgment that needs to be
exercised when writing these signatures, because of the convention in
Haskell where a type may be defined in an ``internal'' module, and then
reexported by a more public module.  Consider these two situations:

\begin{itemize}
    \item The \verb|Map| type is publicly exported by \verb|Data.Map|
    but originally defined in \verb|Data.Map.Internal|.

    \item The strict \verb|ByteString| type is publicly exported by
    \verb|Data.ByteString| but originally defined in
    \verb|Data.ByteString.Internal|.
\end{itemize}
%
We ended up placing \verb|Map| in \verb|Data.Map|,
but \verb|ByteString| in \verb|Data.ByteString.Internal|, as
\verb|binary| used both \verb|Data.ByteString| and \verb|Data.ByteString.Internal|
to manipulate \verb|ByteString|s (we can't define the type in both locations,
since then they won't be the same type!).  In other cases, if a type is exported by
many modules, it may make sense to refactor the imports of the modules
so that we may declare it in a single signature.

Second, the lack of
\emph{recursively dependent signatures}~\cite{crary+:recmod-pldi}
(\ie, signatures that form an import cycle)
sometimes caused problems when a
dependency was implemented using \texttt{hs-boot} files to implement
recursion.  However, this could be easily worked around by adding
a ``synthetic'' signature which collected all of the mutually dependent
data types together, and then have the real signatures reexport
from this signature;  the synthetic signature is then implemented using
a dummy module.  Here is an example of such a synthetic signature
from \texttt{ghc-simple}'s signatures for GHC\@:

\begin{verbatim}
    module RecTypes where

    import ConLike  (ConLike)
    import CoAxiom  (Branched, CoAxiom)
    import OccName  (OccName)

    data Id      -- from Id,   imports Name
    data Name    -- from Name, imports Type
    data TyCon   -- from Type
    data TyThing -- from Type, imports Id
        = AnId Id
        | AConLike ConLike
        | ATyCon TyCon
        | ACoAxiom (CoAxiom Branched)
\end{verbatim}
Without this synthetic signature, the signatures for \texttt{Id}, \texttt{Name},
and \texttt{Type} would form a cycle.

We also have a negative result to report: we were unable to parametrize
the \verb|array| package over \verb|base|.  The difficulty is that
many language features are implemented by selecting a \emph{specific}
implementation from \verb|base|, bypassing the import mechanism.
This means that it's not possible to overload their
implementation merely by changing what module names are in scope.  This
affected a number of things:

\begin{itemize}
    \item Deriving type class instances,
    \item Rebindable syntax (e.g., do notation, integer literals
          and string literals),
    \item Foreign call marshalling, and
    \item Built-in types (e.g., \verb|Integer| for \verb|[0..n]|
          syntax, and booleans for pattern guards).
\end{itemize}
%
As \verb|base| is quite a large package, we think it would be beneficial
if users could write signatures for it; probably the most practical
way forward is to refactor \verb|base| into two packages, one that
contains wired-in entities, and one that does not.

%   \section{A signature for strings}

%   One of the motivating applications of \Backpack{} is handling the
%   so-called ``string problem'', where the Haskell ecosystem has
%   many implementations of string-like data types, and it is difficult
%   to write code that is parametric over the implementation of strings.
%   Indeed, both of the example parametrizations from the previous section
%   are dealing with strings of one form or another.

%   While a package can write a signature for all the functions on strings they
%   require from scratch, it is also useful to be able to reuse signatures
%   between packages.  To this effect, I created a string signature which
%   represented the union of the APIs provided for:

%   \begin{itemize}
%       \item \verb|String| from \verb|base|,
%       \item Strict/lazy ASCII/binary \verb|ByteString| from \verb|bytestring|,
%       \item Strict/lazy UTF-8 \verb|ByteString| from \verb|utf8-string|,
%       \item Strict/lazy \verb|Text| from \verb|text|, and
%       \item \verb|String| from \verb|foundation|
%   \end{itemize}

%   \Red{finish this section}

\section{An alternate package language}
\label{sec:alternative-package-language}

A benefit of having a separation between mixin linking and typechecking
is that it is a simple matter to swap the frontend language with
something else.  To assist in testing Backpack, we implemented a simple
alternate frontend that operates on mixed components (i.e.\ post mixin
linking):

\begin{verbatim}
    unit p where
        dependency base
        signature H where
            data T
        module M where
            import H
    unit q where
        dependency p[H=<H>]
        module N where
            import M
\end{verbatim}
%
The above source code can be run using the \verb|--backpack| major mode
in GHC\@.  This frontend language is used extensively in our test suite
(of over a hundred test cases): in particular, the inline module and
signature syntax is extremely convenient.  This format is not intended
for real world use, since it is difficult to get accurate recompilation
avoidance when source code is all in one file.
